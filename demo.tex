\documentclass[9pt, compress, aspectratio=1610]{beamer}

\usetheme{pl}

\usepackage{booktabs}
\usepackage{minted}

\usepgfplotslibrary{dateplot}

\title{Presentation template}
\subtitle{}
\date{\today}
\author{Timothée Poisot}
\institute{Université de Montréal}

\begin{document}

\maketitle

\section{General informations}

\begin{frame}{Colors}
  \begin{columns}
    \begin{column}[T]{.45\textwidth}
      \textbf{Background colors}
      \vskip 1em
      \begin{tikzpicture}\draw[anchor=west, plFG, fill=plBG, inner sep=0pt](0ex, 0ex) rectangle ++ (1.3ex, 1.3ex);\end{tikzpicture} background (\texttt{plBG})

      \vskip 1em
      \begin{tikzpicture}\draw[anchor=west, plFG, fill=plFG, inner sep=0pt](0ex, 0ex) rectangle ++ (1.3ex, 1.3ex);\end{tikzpicture} foreground (\texttt{plFG})

      \vskip 1em
      \begin{tikzpicture}\draw[anchor=west, plFG, fill=plTX, inner sep=0pt](0ex, 0ex) rectangle ++ (1.3ex, 1.3ex);\end{tikzpicture} text (\texttt{plTX})
    \end{column}\hfil
    \begin{column}[T]{.45\textwidth}
      \textbf{Highlight colors}
      \vskip 1em
      \begin{tikzpicture}\draw[anchor=west, plFG, fill=plLO, inner sep=0pt](0ex, 0ex) rectangle ++ (1.3ex, 1.3ex);\end{tikzpicture} low (\texttt{plLO})

      \vskip 1em
      \begin{tikzpicture}\draw[anchor=west, plFG, fill=plMD, inner sep=0pt](0ex, 0ex) rectangle ++ (1.3ex, 1.3ex);\end{tikzpicture} medium (\texttt{plMD})

      \vskip 1em
      \begin{tikzpicture}\draw[anchor=west, plFG, fill=plHI, inner sep=0pt](0ex, 0ex) rectangle ++ (1.3ex, 1.3ex);\end{tikzpicture} high (\texttt{plHI})
    \end{column}
  \end{columns}
\end{frame}

\begin{frame}{Fonts}

    This documentclass uses the Fira family of fonts. {\sffamily Fira Sans
    is the default font}, and the monospace font is \texttt{Fira Mono}. Math
    support is provided.

\end{frame}

\begin{frame}{Maths}

  $$\int_a^b \alpha \alert{\eta} \approx \sum \prod \epsilon \phi$$

\end{frame}
%--- Next Frame ---%

\begin{frame}[b]{Colors}

  The normal text is set in \texttt{plTX}. The \alert{alerted text} is in
  \texttt{plHI}. It also works for $\alert{\alpha^2}$ maths. This slide has the
  \texttt{b} option to be bottom-aligned, to show the white space at the bottom.

  The structure is in \texttt{plLO}.

\end{frame}
%--- Next Frame ---%

\begin{frame}[t]{Options}

  This example presentation was compiled with \texttt{9pt}, \texttt{compress},
  and \texttt{aspectratio=169}.

  To remove the progress bar below each slide title, use the
  \texttt{noprogressbar} option.

  By the way, the progressbar colors are derived from \texttt{plHI} and
  \texttt{plBG}.

\end{frame}
%--- Next Frame ---%

\section{Plots}

\begin{frame}{PGFplots}

  The theme comes with a few pgfplots options, notably in the form of
  pre-defined cycles.

  Note that there are also some document-level options, such as (notably)
  \texttt{scale only axis}.

  \alert{By design}, most cycles are limited to three or four elements. This
  will force to keep graphs simple.

  The default color of graphics is {\color{plMD}\texttt{plMD}}.

\end{frame}
%--- Next Frame ---%

\begin{frame}{The \texttt{plLineSymbol} cycle}
  \begin{center}
    \begin{tikzpicture}
    \begin{axis}[
    width = 0.8\textwidth,
    height = 0.6\textheight,
    xlabel={Input},
    ylabel={Output},
    grid=major,
    cycle list name = plLineSymbol cycle,
    ]
      \addplot+[samples=21, domain=-1:2,] {0.8*exp(-(x-1.0)^2/(2*0.7^2))};
      \addplot+[samples=21, domain=-1:2,] {0.6*exp(-(x-0.8)^2/(2*0.6^2))};
      \addplot+[samples=21, domain=-1:2,] {0.5*exp(-(x+0.2)^2/(2*0.5^2))};
      \addplot+[samples=21, domain=-1:2,] {0.9*exp(-(x+0.1)^2/(2*0.9^2))};
    \end{axis}
    \end{tikzpicture}
  \end{center}
\end{frame}
%--- Next Frame ---%

\begin{frame}{The \texttt{plLineType} cycle}
  \begin{center}
    \begin{tikzpicture}
    \begin{axis}[
    width = 0.8\textwidth,
    height = 0.6\textheight,
    xlabel={Input},
    ylabel={Output},
    grid=major,
    cycle list name = plLineType cycle,
    ]
      \addplot+[samples=21, domain=-1:2,] {0.8*exp(-(x-1.0)^2/(2*0.7^2))};
      \addplot+[samples=21, domain=-1:2,] {0.6*exp(-(x-0.8)^2/(2*0.6^2))};
      \addplot+[samples=21, domain=-1:2,] {0.5*exp(-(x+0.2)^2/(2*0.5^2))};
      \addplot+[samples=21, domain=-1:2,] {0.9*exp(-(x+0.1)^2/(2*0.9^2))};
    \end{axis}
    \end{tikzpicture}
  \end{center}
\end{frame}
%--- Next Frame ---%

\begin{frame}{The \texttt{plLineColor} cycle (only three values!)}
  \begin{center}
    \begin{tikzpicture}
    \begin{axis}[
    width = 0.8\textwidth,
    height = 0.6\textheight,
    xlabel={Input},
    ylabel={Output},
    grid=major,
    cycle list name = plLineColor cycle,
    ]
      \addplot+[samples=21, domain=-1:2,] {0.8*exp(-(x-1.0)^2/(2*0.7^2))};
      \addplot+[samples=21, domain=-1:2,] {0.6*exp(-(x-0.8)^2/(2*0.6^2))};
      \addplot+[samples=21, domain=-1:2,] {0.5*exp(-(x+0.2)^2/(2*0.5^2))};
    \end{axis}
    \end{tikzpicture}
  \end{center}
\end{frame}
%--- Next Frame ---%

\begin{frame}{The \texttt{plSymbol} cycle}
  \begin{center}
    \begin{tikzpicture}
    \begin{axis}[
    width = 0.8\textwidth,
    height = 0.6\textheight,
    xlabel={Input},
    ylabel={Output},
    grid=major,
    cycle list name = plSymbol cycle,
    ]
      \addplot+[samples=21, domain=-1:2,] {0.8*exp(-(x-1.0)^2/(2*0.7^2))};
      \addplot+[samples=21, domain=-1:2,] {0.6*exp(-(x-0.8)^2/(2*0.6^2))};
      \addplot+[samples=21, domain=-1:2,] {0.5*exp(-(x+0.2)^2/(2*0.5^2))};
      \addplot+[samples=21, domain=-1:2,] {0.9*exp(-(x+0.1)^2/(2*0.9^2))};
    \end{axis}
    \end{tikzpicture}
  \end{center}
\end{frame}
%--- Next Frame ---%

\begin{frame}{The \texttt{plSymbolColor} cycle (only three values!)}
  \begin{center}
    \begin{tikzpicture}
    \begin{axis}[
    width = 0.8\textwidth,
    height = 0.6\textheight,
    xlabel={Input},
    ylabel={Output},
    grid=major,
    cycle list name = plSymbolColor cycle,
    ]
      \addplot+[samples=21, domain=-1:2,] {0.8*exp(-(x-1.0)^2/(2*0.7^2))};
      \addplot+[samples=21, domain=-1:2,] {0.6*exp(-(x-0.8)^2/(2*0.6^2))};
      \addplot+[samples=21, domain=-1:2,] {0.5*exp(-(x+0.2)^2/(2*0.5^2))};
    \end{axis}
    \end{tikzpicture}
  \end{center}
\end{frame}
%--- Next Frame ---%

\begin{frame}{The \texttt{plSymbolFilled} cycle (only three values!)}
  \begin{center}
    \begin{tikzpicture}
    \begin{axis}[
    width = 0.8\textwidth,
    height = 0.6\textheight,
    xlabel={Input},
    ylabel={Output},
    grid=major,
    cycle list name = plSymbolFilled cycle,
    ]
      \addplot+[samples=21, domain=-1:2,] {0.8*exp(-(x-1.0)^2/(2*0.7^2))};
      \addplot+[samples=21, domain=-1:2,] {0.6*exp(-(x-0.8)^2/(2*0.6^2))};
      \addplot+[samples=21, domain=-1:2,] {0.5*exp(-(x+0.2)^2/(2*0.5^2))};
    \end{axis}
    \end{tikzpicture}
  \end{center}
\end{frame}
%--- Next Frame ---%

\begin{frame}{The \texttt{plLineSymbolColor} cycle (only three values!)}
  \begin{center}
    \begin{tikzpicture}
    \begin{axis}[
    width = 0.8\textwidth,
    height = 0.6\textheight,
    xlabel={Input},
    ylabel={Output},
    grid=major,
    cycle list name = plLineSymbolColor cycle,
    ]
      \addplot+[samples=21, domain=-1:2,] {0.8*exp(-(x-1.0)^2/(2*0.7^2))};
      \addplot+[samples=21, domain=-1:2,] {0.6*exp(-(x-0.8)^2/(2*0.6^2))};
      \addplot+[samples=21, domain=-1:2,] {0.5*exp(-(x+0.2)^2/(2*0.5^2))};
    \end{axis}
    \end{tikzpicture}
  \end{center}
\end{frame}
%--- Next Frame ---%

\begin{frame}{The \texttt{plLineSymbolFilled} cycle (only three values!)}
  \begin{center}
    \begin{tikzpicture}
    \begin{axis}[
    width = 0.8\textwidth,
    height = 0.6\textheight,
    xlabel={Input},
    ylabel={Output},
    grid=major,
    cycle list name = plLineSymbolFilled cycle,
    ]
      \addplot+[samples=21, domain=-1:2,] {0.8*exp(-(x-1.0)^2/(2*0.7^2))};
      \addplot+[samples=21, domain=-1:2,] {0.6*exp(-(x-0.8)^2/(2*0.6^2))};
      \addplot+[samples=21, domain=-1:2,] {0.5*exp(-(x+0.2)^2/(2*0.5^2))};
    \end{axis}
    \end{tikzpicture}
  \end{center}
\end{frame}
%--- Next Frame ---%

\begin{frame}{Demonstration with two plots}
  \begin{center}
    \begin{tikzpicture}[baseline]
    \begin{axis}[
    width = 0.4\textwidth,
    height = 0.6\textheight,
    xlabel={Input 1},
    ylabel={Output},
    ymin = 0, ymax = 1,
    xmin = -1, xmax = 2,
    grid=major,
    cycle list name = plLineColor cycle,
    ]
      \addplot+[samples=61, domain=-1:2,] {0.8*exp(-(x-1.0)^2/(2*0.7^2))};
      \addplot+[samples=61, domain=-1:2,] {0.6*exp(-(x-0.8)^2/(2*0.6^2))};
      \addplot+[samples=61, domain=-1:2,] {0.5*exp(-(x+0.2)^2/(2*0.5^2))};
    \end{axis}
    \end{tikzpicture}
    \hfil
    \begin{tikzpicture}[baseline]
    \begin{axis}[
    width = 0.35\textwidth,
    height = 0.6\textheight,
    xlabel={Input 2},
    ylabel={},
    yticklabel pos=right,
    ymin = 0, ymax = 1,
    xmin = -1, xmax = 0,
    grid=major,
    cycle list name = plLineColor cycle,
    ]
      \addplot+[samples=61, domain=-1:2,] {0.3*exp(-(x-1.2)^2/(2*0.6^2))};
      \addplot+[samples=61, domain=-1:2,] {0.7*exp(-(x+0.2)^2/(2*0.8^2))};
      \addplot+[samples=61, domain=-1:2,] {0.9*exp(-(x+0.1)^2/(2*0.4^2))};
    \end{axis}
    \end{tikzpicture}
  \end{center}
\end{frame}
%--- Next Frame ---%

\begin{frame}{The \texttt{plLineGraph} style}
  \begin{columns}
    \begin{column}[c]{.30\textwidth}
      Useful if you don't want to have a full frame around the graph.
    \end{column}\hfil
    \begin{column}[c]{.60\textwidth}
      \begin{center}
        \begin{tikzpicture}
        \begin{axis}[plLineGraph,
        width = 0.8\textwidth,
        height = 0.4\textheight,
        xlabel={Input},
        ylabel={Output},
        grid=major,
        cycle list name = plLineSymbolFilled cycle,
        ]
          \addplot+[samples=21, domain=-1:2,] {0.8*exp(-(x-1.0)^2/(2*0.7^2))};
          \addplot+[samples=21, domain=-1:2,] {0.6*exp(-(x-0.8)^2/(2*0.6^2))};
          \addplot+[samples=21, domain=-1:2,] {0.5*exp(-(x+0.2)^2/(2*0.5^2))};
        \end{axis}
        \end{tikzpicture}
      \end{center}
    \end{column}
  \end{columns}
\end{frame}
%--- Next Frame ---%

\begin{frame}{The \texttt{plCenteredAxes} style}
  \begin{columns}
    \begin{column}[c]{.30\textwidth}
      Useful if you don't want to have a full frame around the graph.
    \end{column}\hfil
    \begin{column}[c]{.60\textwidth}
      \begin{center}
        \begin{tikzpicture}
        \begin{axis}[plCenteredAxes,
        width = 0.8\textwidth,
        height = 0.4\textheight,
        xlabel={Input},
        ylabel={Output},
        grid=major,
        ymin = -1, ymax = 1,
        xmin = -2, xmax = 2,
        cycle list name = plLineColor cycle,
        ]
          \addplot+[samples=21, domain=-1:1,] {0.8*exp(-(x-1.0)^2/(2*0.7^2))-1};
          \addplot+[samples=21, domain=-1:1,] {0.6*exp(-(x-0.8)^2/(2*0.6^2))};
          \addplot+[samples=21, domain=-1:1,] {0.5*exp(-(x+0.2)^2/(2*0.5^2))};
        \end{axis}
        \end{tikzpicture}
      \end{center}
    \end{column}
  \end{columns}
\end{frame}
%--- Next Frame ---%

\begin{frame}{The \texttt{plNoAxes} style}
  \begin{columns}
    \begin{column}[c]{.30\textwidth}

      Used to not draw \alert{any} axes. For whenever you feel like minimalism
      is the way.

    \end{column}\hfil
    \begin{column}[c]{.60\textwidth}
      \begin{center}
        \begin{tikzpicture}
        \begin{axis}[plNoAxes,
        width = 0.8\textwidth,
        height = 0.4\textheight,
        xlabel={Input},
        ylabel={Output},
        grid=major,
        cycle list name = plLineColor cycle,
        ]
          \addplot+[samples=21, domain=-1:2,] {0.8*exp(-(x-1.0)^2/(2*0.7^2))};
          \addplot+[samples=21, domain=-1:2,] {0.6*exp(-(x-0.8)^2/(2*0.6^2))};
          \addplot+[samples=21, domain=-1:2,] {0.5*exp(-(x+0.2)^2/(2*0.5^2))};
        \end{axis}
        \end{tikzpicture}
      \end{center}
    \end{column}
  \end{columns}
\end{frame}
%--- Next Frame ---%

\begin{frame}{The \texttt{plVerticalBarPlot} style}
  \begin{columns}
    \begin{column}[c]{.30\textwidth}

      Sometimes barplots are a necessary evil. We've got you covered. There is a
      \texttt{plBarColors} cycle to go with it. Only three colors.

      Note that the y axis will \alert{always} start at 0 when this style is
      used.

    \end{column}\hfil
    \begin{column}[c]{.60\textwidth}
      \begin{center}
        \begin{tikzpicture}
        \begin{axis}[plVerticalBarPlot,
        width = 0.8\textwidth,
        height = 0.4\textheight,
        xlabel={Input},
        ylabel={Output},
        grid=major,
        cycle list name = plBarColors cycle,
        ]
          \addplot coordinates {(1930,50) (1940,33) (1950,40) (1960,50) (1970,70)};
          \addplot coordinates {(1930,27) (1940,13) (1950,86) (1960,30) (1970,72)};
          \addplot coordinates {(1930,9) (1940,11) (1950,21) (1960,42) (1970,40)};
        \end{axis}
        \end{tikzpicture}
      \end{center}
    \end{column}
  \end{columns}
\end{frame}
%--- Next Frame ---%

\begin{frame}{3D with \texttt{plSurfacePlot}}
  \begin{tikzpicture}
\begin{axis}[plSurfacePlot,
width = 0.5\textwidth,
height = 0.6\textheight,
xlabel={Input},
ylabel={Output},
grid=major,
]
\addplot3[
surf,
domain=-2:2,
domain y=-1.3:1.3,
]
{exp(-x^2-y^2)*x};
\end{axis}
\end{tikzpicture}
\end{frame}
%--- Next Frame ---%

\section{Code}

\begin{frame}[fragile]{General remarks on code}

  The highlighting of code is handled by \texttt{minted}, which uses
  \texttt{pygments} as the syntax engine.

  Slides with code \alert{must} be marked as \texttt{fragile}. For example, the
  beginning of the present slide is:

  \begin{minted}{latex}
    \begin{frame}[fragile]{General remarks on code}
  \end{minted}

  Refer to the \texttt{minted} documentation.

\end{frame}
%--- Next Frame ---%

\begin{frame}[t,fragile]{Example}
  \begin{minted}{python}
    def double(x):
      # This is not a useful function
      return 2 * x + α
  \end{minted}
\end{frame}
%--- Next Frame ---%

\section{Final matters}

\begin{frame}{Lots of original code from}

  \begin{center}\url{github.com/matze/mtheme}\end{center}

\end{frame}


\end{document}
